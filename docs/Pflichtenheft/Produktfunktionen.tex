\chapter{Produktfunktionen}

\section{RION}

\begin{itemize}
	\item[F0110] \textit{Installation:} Der Nutzer kann mit \quotes{rion install \textit{Packagename}} Pakete installieren. Bei der Installation sollen Skripte ausgeführt werden können.
	\item[F0120] \textit{Suchen:} Der Nutzer kann mit \quotes{rion search \textit{text}} Pakete suchen und bekommt eine Liste von passenden Paketen, inklusive einer kurzen Beschreibung derer, ausgeben.
	\item[F0130] \textit{Information:} Der Nutzer bekommt mit \quotes{rion info \textit{Packagename}} detaillierte Informationen zu einem Paket ausgeben.
	\item[F0140] \textit{Aktualisieren:} Der Nutzer kann mit \quotes{rion update} alle installierten Pakete aktualisieren oder er kann mit \quotes{rion update \textit{Packagename}} ein bestimmtes Paket aktualisieren.
	\item[F0150] \textit{Entfernen:} Der Nutzer kann mit \quotes{rion remove \textit{Packagename}} Pakete deinstallieren.
	\item[F0160] \textit{Anleitung:} Der Nutzer erhält durch \quotes{man rion} eine Manpage, in der alle wichtigen Funktionen von RION beschrieben werden.
	\item[F0170] \textit{Installierte Pakete listen:} Der Nutzer erhält durch \quotes{\textit{rion list (Packagename)}} eine Liste aller installierten Pakete, bzw aller installierten Funktionen eines Paketes.
	\item[F0180] Die Repositories auf die RION zugreift soll mit einer config file angepasst werden.
	\item[F0190] RION soll mehrere virtuellen Umgebungen verwalten können. 
	\item[F0111] RION soll mit 	\quotes{rion check (\textit{Packagename (Packageversion)})} überprüfen können, ob bestimmte oder alle installierten Pakete noch korrekt installiert sind.
	\item[F0121] RION soll mit \quotes{rion update} die lokale Datenbank aktualisieren.
\end{itemize}

\section{Endor}

\begin{itemize}
	\item[F0210] \textit{Paket hinzufügen:} Sofern noch keine Version des Paketes in der Datenbank existiert, kann der Nutzer mit \quotes{endor add \textit{Packagename Packagefile}} ein Paket zur Datenbank hinzufügen.
	\item[F0220] \textit{Neue Version hinzufügen:} Der Nutzer kann mit \quotes{endor update \textit{Packagename Packageversion Packagefile}} eine neue Version eines Paketes zur Datenbank hinzufügen.
	\item[F0230] \textit{Beschreibung hinzufügen:} Der Nutzer kann mit \quotes{endor describ \textit{Packagename}} jene Beschreibung zu einem Paket hinzufügen, die beim Suchen durch RION abgerufen wird.
	\item[F0240] \textit{Lizenz:} Der Nutzer kann mit \quotes{rion license \textit{Packagename}} Paketen eine Lizenz hinzufügen.
	\item[F0250] \textit{Paket entfernen:} Der Nutzer kann mit \quotes{endor remove \textit{Packagename}} alle Versionen eines Paketes aus der Datenbank entfernen oder er kann mit \quotes{endor remove \textit{Packagename Versionsnummer}} eine bestimmte Version eines Paketes aus der Datenbank entfernen.
	\item[F0260] \textit{Paket signieren:} Der Nutzer kann mit \quotes{endor sign} alle installierten alle Pakete in der Datenbank signieren oder er kann mit \quotes{rion sign \textit{Packagename (Versionsnummer)}} ein bestimmtes Paket oder auch nur eine bestimmte Version eines Paketes signieren.
	\item[F0270] \textit{Pakete publizieren:} Der Nutzer kann mit \quotes{endor publish \textit{Packagename (Packageversion)}} alle spezifizierten Pakete für RION freischalten.
	\item[F0280] \textit{Pakete sperren:} Der Nutzer kann mit \quotes{\textit{endor unpublish Packagename (Packageversion)}} alle spezifizierten Pakete für RION sperren.
	\item[F0290] \textit{Anleitung:} Der Nutzer erhält durch \quotes{man endor} eine Manpage, in der alle wichtigen Funktionen von Endor beschrieben werden.


\end{itemize}

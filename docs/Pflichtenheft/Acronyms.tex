% This file is a ghost. Only abbreviations for foreign words or new words are stored here. Nothing more

\nomenclature{RION}{Package-Manager und Schnittstelle für die Packetverwaltung zwischen X-FAB-Server und Client}

\nomenclature{X-FAB}{X-FAB ist ein Anbieter für Halbleitertechnologien, welche die Fertigung als auch Desiginunterstützung für Kunden anbietet, die gemischt analog-digitale integrierte Schaltkreise (ICs) entwickeln. X-FAB bietet eine Fülle an unterschiedlichen Technologien für diverse und auch
spezifische Anwendungsmärkte an.}

\nomenclature{INOR}{Serverseitiges BackEnd bei X-FAB}

\nomenclature{Pakete / Packages}{Datenbündel, die als solches eine Funktion haben. (Hiermit sind die Pakete auf den X-FAB Servern gemeint)}

\nomenclature{paru}{Ein AUR-Helper, der es erlaubt unter Arch-Linux und darauf basierenden Distributionen Pakete aus der AUR zu installieren, aktualiesieren, etc.}

\nomenclature{Python}{Programmiersprache: \url{https://www.python.org/}}

\nomenclature{CLI}{Ein \texttt{C}ommand \texttt{L}ine \texttt{I}nterface ist eine Schnittstelle, die es dem Nutzer erlaubt, Textbefehle an ein Programm zu übermitteln. Betriebssysteme, darunter GNU/Linux stellen hierfür eine sogennate Shell, wie zum Beispiel Bash, Dash oder ZSH, zur Verfügung, auf welche man mit einem Terminal zugreifen kann.}

\nomenclature{Manpage}{Eine Hilfeseite für installierte Programme unter Unix-artigen Systemen, auf der, für gewöhnlich, Befehle, Flags und Hinweise zur Benutzung eines Programms verzeichnet sind. Darauf zugriffen wird mit \texttt{man >>name<<} aufrufen.}

\nomenclature{PyPi}{Pypi ist ein Software Sammlung wo Python Packages via pip heruntergeladen werden können}

\nomenclature{pip}{Python Package-Manager: \url{https://pypi.org/project/pip/}}

\nomenclature{PDK}{Ein PDK ist eine komplexe Sammlung von technologiespezifischen Daten für den (aufwändigen und
relativ teuren) Entwurf von anwendungsspezifischen integrierten Schaltkreisen}
\chapter{Vorgehensmodell}
\section{Motivation und Auswahl des Vorgehensmodells}

Als Vorgehensmodell haben wir uns aufgrund der interdisziplinären Zusammensetzung und Größe unseres Teams, sowie der Forderung nach einem agilen Vorgehensmodell seitens der X-FAB für Scrum entschieden.\\
Weiterhin ermöglicht es Scrum durch iteratives Vorgehen flexibel auf Anforderungsänderungen zu reagieren und in jeder Iteration einen funktionsfähigen Prototyp fertigzustellen.\\

Scrum benötigt folgende Rollen innerhalb des Teams:
\begin{itemize}
	\item[Product Owner:] Durch ihn erfolgt eine kontinuierliche Qualitätssicherung der Prototypen. Des Weiteren führt er den Product-Backlog und dokumentiert somit den Gesamtfortschritt des Projekts.

	\item[Scrum Master:] Er moderiert interne Meetings, achtet auf die Einhaltung der Scrum-Methoden, hilft bei der Formulierung der Zielstellungen und unterstützt alle Mitglieder bei aufkommenden Problemen innerhalb ihrer Aufgaben.
	
	\item[Developer:]
	Sind für die Umsetzung der Sprint-Ziele verantwortlich und führen jeweils einen eigenen Sprint-Backlog, in welchem der individuelle Aufgabenfortschritt dokumentiert wird.
	
\end{itemize}

\section{Interne Gruppenorganisation}
Innerhalb unseres Teams fungiert Arndt Schmidt als Scrum-Masterund Valentin Nakov als Product Owner. Als Developer sind Philip Augustin, Calvin Chong Chen Juin, Georg Leander Lehmann, Jonathan Skopp innerhalb des Teams tätig.\\

Organisation der Projektphasen:
\begin{itemize}
	\item[Phase 1:] Organisation der Projektphasen;
	\item[Phase 2:] Beginn der Durchführung der Scrum-Iterationen und somit der Implementierung des Entwurfs inklusive Komponententests.Pro Sprint wird eine Woche angesetzt, die \quotes{Daily Meetings} sind auf Samstag und Dienstag datiert und werden online durchgeführt. Die Planung des kommenden Sprints sowie die Review des abgeschlossenen Sprints finden am Donnerstag statt.
	\item[Phase 3:] Phase 3: Durchführung von Integrations-, Blackbox- und Whitebox- Tests sowie Erstellen der finalen Review Dokumente.
	
	
\end{itemize}
\clearpage
\section{Meilensteine}

\begin{itemize}
	\item 
		Der Package-Manager Rion soll über PyPi (pip) installiert werden können. Das Ziel ist hier, dass jede Linux-Distribution den Manager wie jedes andere bekannte \quotes{Command Line Programm} nutzen kann.
	
	\item Der Packagemanager Rion soll sich erfolgreich \quotes{informieren} können, ob er alle erforderlichen Anforderungen erfüllt und gegebenenfalls interagieren kann. Dazu zählt Folgendes:
		
		\begin{itemize}	
			\item Erstellen von Datenbanken auf Endor und dem lokalen System, um zu prüfen, ob die Packages aktuell sind
			\item Verwalten dieser Datenbanken
			\item Prüfen auf Korrektheit und Vollständigkeit des eigentlichen Package-Managers.
			\item Erstellen und Verwalten von Sicherheitszertifikaten
		\end{itemize}
	\item Der Package-Manager Rion kann Packages herunterladen, installieren, entpacken, verifizieren.
	\item Der Package-Manager Rion kann prüfen (unter zu Hilfenahme der oben genannten Datenbank) ob es Aktualisierungen gibt.
	\item Der Package-Manager Rion kann Abhängigkeiten erkennen und unter den oben genannte Methoden verwalten.
	\item Der Package-Manager Rion muss ein Abbild aller installierten Packages inklusive der daraus resultierenden Abhängigkeiten erstellen können und diese über geeignete Wege ausgeben und auch wieder einlesen.
	\item Der Package-Manager Rion muss fertig gestellt sein.
\end{itemize}